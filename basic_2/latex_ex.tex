\documentclass{article}
\usepackage[utf8]{inputenc}
\usepackage[french]{babel}
\usepackage{amsmath}

\title{Exemple de Document Structuré}
\author{IA PDF Learning}
\date{\today}

\begin{document}

\maketitle

\section{Introduction}
Ce document est un exemple de fichier LaTeX structuré avec des titres et des sous-titres. Il montre comment organiser un document académique ou éducatif.

\subsection{Objectif}
L'objectif de cet exemple est de démontrer comment utiliser LaTeX pour créer un document bien structuré, avec des sections et des sous-sections.

\section{Contenu Principal}
\subsection{Exemple de Sous-Section}
Voici un exemple de sous-section. Vous pouvez inclure du texte, des équations, et d'autres éléments ici.

\subsubsection{Équations Mathématiques}
Par exemple, l'équation d'une ligne droite est donnée par :
\[ y = mx + b \]
où \( m \) est la pente et \( b \) est l'ordonnée à l'origine.

\subsection{Autre Sous-Section}
Cette sous-section peut contenir d'autres informations pertinentes pour votre document.

\section{Conclusion}
En conclusion, cet exemple montre comment structurer un document LaTeX avec des titres et des sous-titres. Cela peut être très utile pour organiser des documents académiques ou éducatifs de manière claire et professionnelle.

\end{document}
